
\subsection{Discrete Uniform Distribution}

\ssn{Equally likely outcomes}
Consider a sample space $S_n$ which has $n$ elements. Let $\PP(x) = 1/n$ for all $x \in S_n$, so that all outcomes of the experiment are equally likely.   By the discussion in \S\ref{prbfn} this defines a probability mass function on $S_n$. Let  $A \subseteq S_n$ be an event.  
\tcb
\[
  \text{For ``equally likely outcomes'' we have}\quad\quad \PP(A) = \frac{\# A}{\# S}
\]
where we write $\# A$ for the number of elements in the set $A$, etc.
\etcb
\end{n} 

\ssn{Definition: uniformly random}
When we have an experiment with an ``equally likely outcomes'' probability function, we often say that the outcome is  \ul{uniformly random}.  

Thus, for example, rolling a D6 is equivalent to choosing a natural number less than or equal to six ``uniformly randomly''. 
\end{n}

\ssn{Examples}
\begin{enumerate}
    \item Rolling a die has six equally likely outcomes. What is the probability that we roll an even number? Let $S = \{ 1,2,3,4,5,6\}$ and let $A = \{ 2,4,6 \}$.  Then 
    \[
       \PP(A) = \frac{\# A}{\# S} = \frac36 = \frac12. 
    \]
 \item I toss a coin four times. What is the probability I get three heads and one tail?   Here $S$ is the set of all sequences of four terms, each of which can be `H' or `T'. Every such sequence is equally likely.  There are 16 such sequences so $\# S = 16$.  The event we want is 
  \[
      A = \{ (T,H,H,H), (H,T,H,H), (H,H,T,H), (H,H,H,T) \}. 
  \]
 So the probability is $\PP(A) = {\# A}/{\# S} = 1/4$. 
\end{enumerate}
\end{n}


\sse{}  I toss a coin three times. What is the probability that I get: (A) three heads; (B) precisely two heads and (C) at most one head? 
\end{e} 

\sss 
The sample space consists in this case of all eight possible sequences of heads and tails and \emph{all eight outcomes are equally likely}:  
\[
 S = \{ TTT, TTH, THT, THH, HTT, HTH, HHT, HHH \}.  
 \]
Precisely one of the eight possibilities has three heads and so the answer to (A) is $1/8$.  There are three possibilities with precisely two heads and so the answer to (B) is $3/8$.  For (C), the relevant elements of $S$ are TTT, TTH, THT, HTT.   Thus the probability is $4/8 = 1/2$. 
\end{s}

\sse
What is the probability of getting more heads than tails if I toss a coin four times.  And (without doing a complicated calculation) what is the probability if I toss it 19 times?  
\end{e}
\sss 
The sample space now has 16 elements. There is one element HHHH which is all heads and four possibilities with three heads and one tail. (Think where the tail is.) So the probability is $5/16$.  If I toss 19 times, there cannot be equal numbers of H and T.  By symmetry, ``more H'' and ``more T'' have to be equally likely and so the probability is $1/2$. 
\end{s}

\ssn{Using tables} 
Sometimes a table can simplify equally likely outcome arguments. Consider the following. 

I have two six-sided dice, one black, one white, with non-standard labelling. (Black) is labelled with five `3's and a single `6'.  (White) is labelled with two `1's, one `4' and three `5's.     If both are rolled, what is the probability that (black) beats (white)?

Let us take as sample space all $6^2=36$ different ways two dice can land.  Since there are five `3's on the red die and two `1's on the blue,  $5 \times 2 = 10$ of the 36 equally likely rolls come out as a `3' and a `1'.  Computing the other outcomes, we get 
\begin{center}
\begin{tabular}{|c|ccc|}
 \hline 
   & White 1 & White 4 &  White 5 \\ 
  \hline
 Black 3 & 10 & 5 & 15   \\
 Black 6 & 2 & 1  & 3  \\
 \hline
\end{tabular}
\end{center}
Notice the numbers in the table add up to $36$ as they should.  Black wins in the top left entry and all the bottom row.  Thus the probability of a black win is $(10+2+1+3)/36 = 4/9$. 
\end{n}

\sse{}
If I roll the Black die in the previous example against a standard D6, what is the probability of Black winning, Black losing and of the two rolls being equal? 
\end{e}
