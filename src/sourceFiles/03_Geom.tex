%\ssn{Learning Outcomes}
%After studying this week you will be able to:
%\begin{itemize}
%\item use the geometric distribution to calculate probabilities;
%\item explain what is meant by a continuous random variable and how it is represented by a probability density function;
%\item compute probabilities and expectations for continuous random variables;
%\item model appropriate problems using exponential or uniform distributions.
%\end{itemize}
%\end{n}

\subsection{Geometric Distribution}

\ssn{Motivating problem} 
I have a process which, independently I execute it, has a probability $p$ of success. Let $X$ be the number of tries up to and including my first success. What is the probability that $\PP(X=k)$, i.e.\ I succeed for the first time on my $k$-th try. 

Setting $q=1-p$, to succeed on the $k$-th attempt, I have to fail $k-1$ times in a row and then succeed. Since the tries are all independent, we have 
 \[
         \PP(X=k) = q^{k-1} p. 
  \] 
\end{n}

\ssn{Definition: Geometric random variable} 
A random variable $X$ is \ul{geometric} and we write $X \sim \mathop{Geom}(p)$ if \hfill 
 \tcb 
   \[
         \PP(X=k) = q^{k-1} p \quad \text{where $q=1-p$.} 
  \] 
 \etcb
\end{n} 

\ssn{Proposition}  The expected value of $X \sim  \mathop{Geom}(p)$ is $\EE(X) = 1/p$. 
\begin{proof}
From the definition of $\EE(X)$ we have 
 \[
     \EE(X) = p + 2pq + 3pq^2 + 4 pq^3 + \dots .
 \]
Multiplying by $q$ we have 
 \[
      q \EE(X) = pq + 2pq^2 + 3pq^3 +  \dots .
 \]
 Subtracting
  \[
  (1-q) \EE(X) = p \EE(X) =  p + pq + pq^2 + pq^3 + \dots = \frac{p}{1-q} = 1.
  \]
 where we have summed the GP and used $q=1-p$. 
\end{proof}
\end{n}

\ssn{Example} 
 Suppose I roll a pair of D6 repeatedly I and count the number of rolls until I roll a total of `5'.  The probability of rolling a total of five is $4/36 = 1/9$. 
 
 So with $X \sim \mathop{Geom}(1/9)$ we have that the probability of succeeding first on the third roll is 
  \[
       \PP(X = 3) = q^2 p = \left( \frac89 \right)^2   \frac19 = 
       \frac{64}{729}.
  \]
  The expected number of rolls is $\EE(X) = 1/p = 9$. 
\end{n} 

\sse{Exercise} 
One day each week I go fishing in a lake and  I independently have a probability of $1/52$ of catching the big carp that lives there.
Let $X=k$ if I first catch the carp on my $k$-th day fishing. 
\begin{enumerate}
    \item On which day of fishing am I most likely to catch the carp? 
    \item What is the probability that I first catch the carp on my third day fishing? 
    \item What is the expected number of fishing days until I catch the carp? 
\end{enumerate}
\end{e}

\sss
Here $X \sim \mathop{Geom}(1/52)$. The value of $\PP(X=k)$ is greatest when $k=1$ and so the first day is the most probable day that I catch the fish.  The probability of first catching it on day 3 is $(51/52)^2 (1/52)$.  The expected number of days is $1/p$ = $52$. 
\end{s} 
