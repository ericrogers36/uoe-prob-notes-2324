

\ssn{Indicator functions}  Given a subset $A \subseteq S$ of a sample space, we define its \emph{indicator function} $I_A$ which is the function on $S$ defined by 
 \[
    I_A(x) = \begin{cases}   1 & x \in A \\ 0 & x \not\in A.
     \end{cases} 
 \]
\end{n}

\ssn{Use of indicator functions} 
We can think of $I_A$ as being a random variable: it takes the value $1$ if the event happens and $0$ if it does not.   So,
 \[
 \EE(I_A) = \PP(A).
 \]
With this trick we can turn many probability problems into random variable problems.  
\end{n}

\ssn{Properties} Let $A,B$ be subsets of a sample space $S$. 
\begin{enumerate}
\item $I_A^2 = I_A$ (and more generally $I_A^k = I_A$ when $k \in \mathbb{N}$). 
\item $I_{A^c} = 1 - I_A$
\item $I_{A\cap B} = I_A I_B$ (and in particular if $A \subseteq B$ then 
$I_A I_B = I_A$).
\end{enumerate}
\end{n}

\ssn{The general inclusion-exclusion principle} \label{giep}
Writing the general version is fairly intimidating: 
\[
 \PP(A_1 \cup A_2 \cup\dots\cup A_n) =
  \sum_{k=1}^n  (-1)^{k+1}  
 \sum_{1\leq i_1 < i_2 <\dots < i_k \leq n}
\PP( A_{i_1} \cap A_{i_2} \cap \dots \cap A_{i_k}  )
\]
\begin{proof} 
We give a proof for the $n=3$ case using indicator functions. 

Now suppose $M = A \cup B \cup \dots \cup C$. Then the function 
 \[
   (I_M - I_A) (I_M - I_B) (I_M - I_C) = 0 
  \]
 because for $x \not\in M$ all the terms vanish and for $x \in M$ we know $x$ is in at least one of $A,B,C$ and the corresponding term vanishes. 
 Multiplying out and rearranging:
  \[
    I_M^3 = I_M^2 (I_A+I_B+I_C) - I_M (I_A I_B + I_A I_C + I_B I_C 0 
       + I_A I_B I_C. 
  \]
Now, $I_MI_A = I_A$ since $A$ is a subset of $M$ and similarly for $B$ and $C$. So the ``$I_M$''s on the right-hand side can be dropped. And on the left we can replace $I_M^3$ by $I_M$.  Then 
 \[
  I_M = (I_A+I_B+I_C) - ( I_{A\cap B} + I_{A \cap C} + I_{B \cap C}) 
    + I_{A \cap B \cap C}.
\]
Taking expectations of both sides (they are random variables) and using the linearity from \ref{exlin} we arrive at
 \[
  \PP(M) = (\PP(A) + \PP(B) + \PP(C) ) - 
   (\PP(A \cap B) + \PP(A\cap C) + \PP(B \cap C) ) 
    + \PP( A \cap B \cap C). 
 \]
\end{proof}

\end{n}
