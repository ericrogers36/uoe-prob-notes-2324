\section{Foundations of probability, inclusion-exclusion and equally likely outcomes}

\subsection{Basic Ideas}
If we toss a fair coin many times then we expect about half of the tosses will result in ``heads'' (hereafter abbreviated as `H'). We say
\begin{quotation}
   ``the probability of the result `heads' when tossing a fair coin is $\PP(H) = 1/2$''. 
\end{quotation}
And of course the probability of ``tails'' $(T)$ must also be one half: $\PP(T)=1/2$.

Similarly, when rolling a fair $6$-sided die (hereafter abbreviated to ``D6'') with faces $1,2,\dots,6$, the probability of obtaining a $6$ is equal to $\PP (6) = 1/6$ (as indeed it is for any of the other five possible outcomes).

We could ask also what is the probability that on rolling the die the result is a perfect square.  The only perfect squares on the die are $1$ and $4$ and so two of the six possibilities are perfect squares. Thus the probability is $1/3$.

\tcb
We will be using $\PP$ a lot for the function that takes some ``event'' and evaluates its probability. Use a big, beautiful, ``blackboard bold'' $\PP$ in your notes and hand-ins, etc. It's easier to do maths right with a nice, clear notation. 

\vspace*{-5.5ex}
{\LARGE \[ \PP( \text{``blah''}) = \text{the probability that ``blah'' happens} \] } 
\etcb

\subsection{Mathematical Formulation}
\begin{defn}[\textbf{State Space / Universal Set}]
We denote the \emph{state space} or \emph{universal set} by $\Omega$ or $S$. 
\end{defn}
\begin{defn}[\textbf{Event}]
An \emph{event} is a "nice" subset of the state space, i.e. an event $A$ fulfils $A \subseteq \Omega$.
The set of all events often is denoted by $\mathcal{A}$, $\mathcal{E}$ or $\mathcal{F}$.
\end{defn}
\begin{defn}[\textbf{Probability Measure}]
A \emph{probability measure} $\PP$ is a map taking events as argument with
\begin{enumerate}
\item $\PP: \mathcal{A} \to [0,1]$,
\item $\PP(\Omega)=1$,
\item For a countable index set $I$ with $(A_i)_{i \in I}$ being disjoint events we have $\PP( \bigsqcup_{i \in I} A_i) = \sum_{i\in I} \PP(A_i)$.
\end{enumerate}
\end{defn}
\subsection{Discrete Uniform Distribution}
\begin{defn}[\textbf{Discrete Uniform Distribution}]
Let $\Omega$ be non-empty and finite and $\PP(\{\omega\})= \frac{1}{\vert \Omega \vert }$ for all $\omega \in \Omega$.  Then
\[ \PP(A) = \frac{ \vert A \vert }{ \vert \Omega \vert }. \]
We call this setting a \underline{discrete uniform distribution} or \underline{uniformly random}.
\end{defn}

\subsection{Inclusion-Exclusion Principle}

\begin{thm}[\textbf{Inclusion-Exclusion Principle for 2 sets}]
Let $A$ and $B$ be two events. Then
\[\PP(A \cup B) = \PP(A) + \PP(B) - \PP(A \cap B).\]
\end{thm}
\begin{proof}
Observe that $A \cup B = (A \setminus B) \sqcup (A \cap B) \sqcup (B \setminus A)$, which is a disjoint union. Hence,
\begin{align*}
\PP(&A \cup B) 
\\ 
&= \PP( (A \setminus B) \sqcup (A \cap B) \sqcup (B \setminus A) )
\\&
= \PP(A \setminus B) + \PP(A \cap B) + \PP(B \setminus A)
\\&
= \PP(A \setminus B) + \PP(A \cap B) + \PP(A \cap B) + \PP(B \setminus A) - \PP(A \cap B)
\\&
= \PP((A \setminus B) \sqcup (A \cap B) ) + \PP( (A \cap B) \sqcup (B \setminus A)) - \PP( A \cap B)
\\&
= \PP( A ) + \PP(B) - \PP( A \cap B).
\end{align*}
\end{proof}
\noindent This can be generalised as follows.
\begin{thm}
Let $n \in \NN$ and $A_i$, $i\in \{ 1, \ldots, n \}$ be events. Then
\[\PP\left(\bigcup_{i=1}^n A_i \right) = \sum_{k=1}^n (-1)^{k+1} \sum\limits_{1 \leq i_1 < \ldots < i_k \leq n} \PP\left( \bigcap_{j=1}^k A_{i_j}\right).\]
\end{thm}
\begin{proof}
See workshop of Week 1.
\end{proof}