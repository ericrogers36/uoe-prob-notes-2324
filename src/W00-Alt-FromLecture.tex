\section{Motivation / Why study probability?} 

We talk about it all the time. How likely is it that somebody trying numbers at random will get my PIN right? What's the chances of Chelsea winning the premier league this year?  

Probability is the mathematical theory of chance. It underlies the theory of statistics (which is about making valid deductions from data) but is an active mathematical subject in its own right with links to the theory of integration, number theory and other areas. We will for example work out how likely it is that two large numbers chosen at random (whatever that means) have greatest common divisor equal to one. 

Initially, most of the probabilistic theory was motivated by gamblers. After all, they wanted to know ``if I play this game, how likely is it for me to win and how much money will I make''? Or, on the other hand, ``how likely is it that I will lose and how much will I lose if I play it". 

In the early days of probabilistic theory, there wasn't a clear approach - later on there will be examples of this, where people argued one thing, and in reality it turned out to be completely different. What gamblers also did (and some people still do when they go into the pub and play dice) is to take notes on how often something has occurred. This is a statistical way of tracking how probable something is, however until Kolmogorov introduced probability axioms, things were a bit iffy. 

