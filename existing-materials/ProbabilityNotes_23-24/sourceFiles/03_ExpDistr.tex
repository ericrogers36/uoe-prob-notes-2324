
\subsection{Exponential Distribution}

\ssn{Example: back to the radioactive atom} (Do not get bogged down in the following argument if you find parts of it hard. The main point is that at the end we derive a pdf for the situation. The derivation is not examinable.)  

Let us return to the radioactive atom and see if we can find a candidate for a pdf. We will assume there is a constant $\lambda >0$ such that \emph{if the atom is intact at time $t$} then it decays in the interval $[t, t + \Delta t]$ with probability $\lambda \Delta t$ (in the limit as $\Delta t \map 0$).

Now, letting $T$ be the RV which is when the atom decays and letting its pdf be $f_T$ we have the following as $\Delta t\map 0$.
\begin{eqnarray*}
  f_T(t) \Delta t &=& \PP (t \leq T \leq t+\Delta t) \\
   &=& \PP(\text{intact at time $t$}) \;\PP( \text{decays in $[t,t+\Delta t] \st$ intact at time $t$})  \\
   &=& \PP(\text{intact at time $t$}) \lambda \Delta t  \\
   &=& (1- \PP( 0 \leq T \leq t))  \lambda \Delta t.  
\end{eqnarray*}
Dividing be $\Delta t$ and replacing the probability by an integral of the pdf we have
\[
   f_T(t) = \lambda \left( 1 - \int_0^t f_T(u) \dd u \right). 
\]
Differentiating, we have 
 \[
   f_T'(t) = -\lambda f_T(t) \text{ and so } f_T(t) = A e^{-\lambda t} \text{ for some $A>0$}.  
 \]
For $\int_0^\infty f_T(t) \dd t = 1$ we need $A=\lambda$ and so finally we have 
\[
     f_T(t) = \lambda e^{-\lambda t} ,\quad t \geq 0. 
\]
\end{n}

\ssn{Definition} \hfill 
\tcb 
A random variable $X$ on $[0,\infty]$ with pdf of the form 
 \[ 
 f_X(x) = \lambda e^{-\lambda x} \text{ (where } \lambda >0
  \]
  is a constant) is called an \emph{exponential random variable}.  
\etcb 
So the sample space is $S = [a,b]$. It does not matter whether we take an open interval $(a,b)$ instead.  
\end{n}

\sse 
\begin{enumerate}[(a)]
\item Show that the probability of decay occurring between the times $t=0$ and $t=1/\lambda$ is $(e-1)/e \approx 0.632$.
\item The \emph{half life} $\tau$ of a radioactive atom is the time at which there is a 50\% chance that an atom that was intact at $t=0$ will have decayed.   Show that $\tau = \ln 2 / \lambda$. 
\end{enumerate}
\end{e}

\sss
\begin{enumerate}[(a)]
\item  We compute 
  \[
     \int_0^{1/\lambda} \lambda e^{-\lambda t} \dd t = [ 1 - e^{-\lambda t}]_0^{1/\lambda} = \frac{e-1}{e}. 
  \]
 \item To find $\tau$ such that the probability of decay is $1/2$ we solve   
 \[
   [ 1-e^{-\lambda}]_0^\tau = \frac12  
 \]
to get $e^{-\lambda\tau}= 1/2$ or $\lambda\tau = \ln 2$. 
\end{enumerate}
\end{s}



\sse{} In a model of a call centre, at a given moment, the time in seconds until the next call arrives is a random variable $T$ that has a pdf of the form 
 \[
    f_T(t) =  k e^{-t/8}, \qquad \text{where $t \in [0,\infty)$ and $k$ is a constant to be determined.}
 \] 
What is the probability that a call arrives in the first four seconds?   (Answers: $k=1/8$ comparing with the general exponential distribution. The integral for the probability should evaluate to approx 0.3935.) 
 \end{e}

\sse{}
The random variable $X$ defined on the interval $[0,1]$ has pdf given by $f_X(x) = k \sqrt{1-x^2}$.  Find the value for $k$. No explicit integration should be necessary! 
(You should arrive at $k=4/\pi$.) 
\end{e}

\sss
Notice that the graph of $y = \sqrt{1-x^2}$ is a quarter of a unit circle so the area under it is $\pi/4$.  Alternatively, integrate!
\end{s}


\ssn{Convention} 
Henceforth we will assume that for a continuous random variables $X$, the pdf $f_X(x)$ is defined for the whole real line. If $X$ takes values only in an interval $(a,b)$ we simply take $f_X(x) =0$ for all $x$ outside $(a,b)$. 
\end{n}
