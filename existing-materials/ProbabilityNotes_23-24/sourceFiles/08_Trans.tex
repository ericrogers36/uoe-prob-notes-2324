\subsection{Transformations}

\ssn{Motivating problem}

You are working in a laboratory and know from experience that the result $X$ of some experiment is uniformly distributed in the range $[1,2]$. For a public presentation you want to do this experiment in a larger scale. You plan to triple the size, i.e.\ the new result $Y$ will be $Y=3 \cdot X$. To implement some safety precautions you need to know the distribution of $Y$. \\
What is the pdf of $Y$ here?

\end{n}

\ssn{Example}

Let $X \sim \unif(1,2)$ and $Y:= 3 \cdot X$. What is the pdf $f_X$?

We know from Proposition \ref{PropCdfPdf} that $f_Y(y) = \partial_y F_Y (y)$, hence we can just concentrate on finding $F_Y$.
First, we have a look at the definition $F_Y(y) = \PP (Y \leq y)$. Here we notice, that since $Y= 3X$, we can simply replace it to obtain $\PP(Y \leq y) = \PP(3X \leq y)$. Doing the equivalent transform of dividing both sides of the inequality by 3 we have $\PP(3X\leq y)= \PP(X \leq y/3)$. Notice that this is the cdf of $X$, which we know to be
$$F_X(x)= \int_{-\infty}^x \1_{\{x \in [1,2]\}} dx
= \left\{ \begin{array}{ll}
0,\quad & x < 1 \\
x-1, \quad & 1 \leq x \leq 2 \\
1, \quad & 2 < x.
\end{array}
\right.$$
Putting this together we have
\begin{align*}
F_Y(y)
&
= \PP(Y\leq y)
%\\ &
= \PP(3X \leq y)
%\\ &
= \PP \Big(X \leq \frac{y}{3} \Big)
%\\ &
= F_X \Big(\frac{y}{3} \Big)
\\ &
= \left\{ \begin{array}{ll}
0,\quad & y/3 < 1 \\
y/3-1, \quad & 1 \leq y/3 \leq 2 \\
1, \quad & 2 < y/3.
\end{array}
\right. \quad
%\\ &
= \left\{ \begin{array}{ll}
0,\quad & y < 3 \\
y/3-1, \quad & 3 \leq y \leq 6 \\
1, \quad & 6 < y.
\end{array}
\right.
\end{align*}
Differentiating, we get $f_Y(y) = \partial_y F_Y(y) = \frac13 \1_{\{ y \in [3,6] \} }$, i.e.\ $Y$ is uniformly distributed on $[3,6]$.

\


\end{n}

\ssn{Theorem}


We did the last example for a quite simple transformation and a quite simple distribution. Can we do this in general?

Let us assume that now we are given the pdf of $X$, denoted by $f_X$ and that $Y:= g(X)$ for some function $g$. We can say the following.

\tcb
Let $X$ be a continuous random variable with given pdf $f_X$ and let $g: \RR \to \RR$ be such that its inverse $g^{-1}$ exists (i.e.\ there exists a function $g^{-1}$ such that $g^{-1}(g(x)) = x$), is differentiable and monotone increasing where $f_X$ is positive. Define $Y$ as $Y:= g(X)$. Then
$$f_Y(y) 
= f_X(g^{-1}(y))  \cdot (g^{-1})'(y).
$$
\etcb

\begin{proof}
We do the exact same calculation as above. In detail it is
\begin{align*}
F_Y(y)
&
= \PP(Y\leq y)
%\\ &
= \PP(g(X) \leq y)
%\\ &
= \PP (g^{-1}(g(X)) \leq g^{-1}(y))
%\\ &
= \PP (X \leq g^{-1}(y))
%\\ &
= F_X (g^{-1}(y)),
\end{align*}
where we used that applying the inverse of $g$ on both sides gives an equivalent statement. Finally, we differentiate to obtain
\begin{align*}
f_Y(y)
&
= \partial_y F_Y(y)
\\&
= \partial_y F_X( g^{-1}(y))
\\ &
= (F_X)'(g^{-1}(y)) \cdot (g^{-1})'(y)
\\ &
= f_X(g^{-1}(y)) \cdot (g^{-1})'(y),
\end{align*}
where the chain rule was applied.
\end{proof}

Now let us have a look at two examples where $g$ does not have an increasing inverse.

\end{n}

\ssn{Example}

Let $X \sim Exp(\lambda)$, $g(x)= -a x$ for some $a > 0$ and $Y:=g(X)$.
\\
We can still use the basic calculation from above. However, the step where we formally applied the inverse of $g$ requires closer attention.
\begin{align*}
F_Y(y)
&
= \PP(Y\leq y)
%\\ &
= \PP(-aX \leq y)
\end{align*}
Here it becomes obvious that the calculation does not go the exact same way. Instead we obtain
$$ \PP(-aX \leq y) = \PP\Big(X \geq -\frac{y}{a}\Big).$$
Continuing our calculation with this slightly changed step and using that single points have probability 0, we get
\begin{align*}
F_Y(y)
&
=\PP\Big(X \geq -\frac{y}{a}\Big)
%\\&
= 1 - \PP\Big(X < -\frac{y}{a}\Big)
= 1 - \PP\Big(X < -\frac{y}{a}\Big)
= 1 - F_X\Big( -\frac{y}{a}\Big)
.
\end{align*}
Now we again can differentiate to obtain that
\begin{align*}
f_Y(y)
&= \partial_y F_Y(y)
= \partial_y \Big(1-F_X\Big(-\frac{y}{a}\Big)\Big)
= \frac1a f_X\Big(-\frac{y}{a}\Big)
= \left\{ \begin{array}{ll}
0, \quad & -\frac{y}{a} < 0 \\
\frac1a \lambda e^{-\lambda (-\frac{y}{a})}, \ & 0 \leq -\frac{y}{a}
\end{array} \right.
\\ &
= \left\{ \begin{array}{ll}
0, \quad & y > 0 \\
\frac{\lambda}{a} e^{\frac{\lambda}{a} y}, \ & 0 > y.
\end{array}
\right.
\end{align*}
Note that this is not an exponential distribution, since its mass is on the negative axes.

\end{n}

\ssn{Example}

As a final example, we have a look at what happens, if $g$ is not invertible. For this, let $g(x)= x^2$, $X \sim \unif([-1,2])$ and $Y:= g(X)$, i.e.\ $Y:= X^2$ and $F_X(x) = \frac13 (1+x) \1_{\{x \in [-1,2]\}} + \1_{\{x > 2\}}$. Starting as usual, we have
$F_Y(y)
= \PP ( X^2 \leq y ).$ Here we see the problem, that we cannot invert, since left from $0$ the inverse is $-\sqrt{\cdot}$, while on the right it is $\sqrt{\cdot}$. Instead, we find the equivalent formulation $\vert X \vert \leq \sqrt{y}$. However, we need to require for this that $y \geq 0$. Because $X^2 \geq 0$, this is no problem and just means that $F_Y(y) =0$ for all $y<0$. Collecting all of that, we progress to
$$ F_Y(y) 
= \PP ( X^2 \leq y ) 
= \PP ( \vert X \vert \leq \sqrt{y}) \1_{\{y \geq 0\}}
= \PP ( -\sqrt{y} \leq X \leq \sqrt{y}) \1_{\{y \geq 0\}}
.
$$
Replacing the probability with the cdf this reads as
$$ F_Y(y)
= \PP ( -\sqrt{y} \leq X \leq \sqrt{y}) \1_{\{y \geq 0\}}
= (F_X(\sqrt{y})-F_X(-\sqrt{y}))\1_{\{y \geq 0\}}
.
$$
Now we can use the explicit expression for $F_X$ from above to get
\begin{align*}
F_Y(y)
&
=
\left[ \frac13 (1+\sqrt{y}) \1_{\{\sqrt{y} \in [-1,2]\}} + \1_{\{\sqrt{y} > 2\}} - \frac13 (1-\sqrt{y}) \1_{\{-\sqrt{y} \in [-1,2]\}} - \1_{\{-\sqrt{y} > 2\}} \right] \1_{\{y \geq 0\}}
,
\end{align*}
which looks quite intimidating. However, in a first step we can simplify it, by removing all cases where $\sqrt{y}$ would have to be negative and as a second step, we split it up into the disjoint cases $\sqrt{y} \in [0,1]$, $\sqrt{y} \in [1,2]$ and $\sqrt{y} > 2$. Thus, we have
\begin{align*}
F_Y(y)
&
=
\left[ \frac13 (1+\sqrt{y}) \1_{\{\sqrt{y} \in [0,2]\}} + \1_{\{\sqrt{y} > 2\}} + \frac13 (\sqrt{y}-1) \1_{\{\sqrt{y} \in [0,1]\}} - 0 \right] \1_{\{y \geq 0\}}
\\ &
=
\left\{ \begin{array}{ll}
0, \quad & y < 0 \\
\frac13 (1+\sqrt{y}) + \frac13 (\sqrt{y}-1), \quad & 0 \leq \sqrt{y} \leq 1\\
\frac13 (1+\sqrt{y}), \quad & 1 < \sqrt{y} \leq 2 \\
1, \quad & 2 < \sqrt{y}
\end{array}
\right.
\\ &
=
\left\{ \begin{array}{ll}
0, \quad & y < 0 \\
\frac23 \sqrt{y}, \quad & 0 \leq y \leq 1\\
\frac13 (1+\sqrt{y}), \quad & 1 < y \leq 4 \\
1, \quad & 4 < y
\end{array}
\right.
.
\end{align*}
A quick sanity check is, whether this is actually a non-decreasing function as it should be. Since every single part is non-decreasing, we only need to have a look at the border cases $y=0,1,4$ which all turn out to be fine.

So finally, we only have to differentiate to obtain $f_Y$. Doing so we get
\begin{align*}
f_Y(y)
= \partial_y F_Y(y)
= \left\{ \begin{array}{ll}
0, \quad & y < 0 \\
\frac{1}{3 \sqrt{y}}, \quad & 0 \leq y \leq 1\\
\frac{1}{6 \sqrt{y}}, \quad & 1 < y \leq 4 \\
0, \quad & 4 < y
\end{array}
\right.
.
\end{align*}

\end{n}


\subsection{Exercises and problems}

\sse 
The random variable $X \sim \gamma(2,\lambda)$ and $Y:= \ln (X)$. Find the pdf $f_Y$. \\ (Ans.: $f_Y(y) = \frac{\lambda^2}{2} e^y e^{-\lambda e^y} \1_{\{y > 0\}}$ )
\end{e}

\sse{}
Show that for any continuous random variable $X$ it is true that $F_X(X) \sim \unif(0,1)$. 
\end{e}

\ssp{}
Let $X$ have any continuous distribution and let $Y \sim \unif(0,1)$. Show that $Z:= F_X^{-1} (Y)$ has the same distribution as $X$, i.e.\ that $F_Z = F_X$.
\end{e}

\sse{}
Let $X \sim \expo(\lambda)$.  Compute $F_Y$ for $Y:= X^2$.   (Ans: $(1-e^{-\lambda \sqrt{y}}) \1_{\{ y \geq 0 \}}$) 
\end{e}

